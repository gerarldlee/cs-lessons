\documentclass[12pt]{article}

\usepackage{url}
\usepackage{fullpage}
\usepackage{amssymb,amsfonts,amsmath}
\usepackage{graphicx}

\newcommand{\eps}{\varepsilon}
\newcommand{\R}{\mathbb{R}}
\newcommand{\ceil}[1]{\lceil #1 \rceil}

\begin{document}

\thispagestyle{empty}

\begin{center}
{\Large \textsc{CS 125 Algorithms \& Complexity} --- Fall 2016}

\bigskip

{\Large \textsc{Problem Set 6}}

\smallskip

Due: 11:59pm, Friday, October 21st

\bigskip

{\footnotesize See homework submission instructions at \url{http://seas.harvard.edu/~cs125/fall16/schedule.htm}}
\end{center}

\textbf{Problem 5 is worth one-third of this problem set, and problems 1-4 constitute the remaining two-thirds.}

\bigskip

\section*{Problem 1} 

For each of the following languages, determine whether or not they are regular and prove your answer.
\begin{itemize}
\item[(a)] (2.5 points) $\{ w\in \{a,b\}^* : \mbox{$w$ has more $a$'s than $b$'s}\}$.
\item[(b)] (2.5 points) $\{ w\in \{a,b\}^* : \mbox{the number of occurrences of $ab$ in $w$ equals the number of}$
\item[] \hspace{1in}$\mbox{occurrences of $ba$}\}$.
\item[(c)] (2.5 points) $\{ w\in \{a,b,\ldots,z\}^* : \mbox{$|w|$ is a perfect square}\}$.
\item[(d)] (2.5 points)  $\{ w\in \{0,1\}^* : \mbox{$w$ is the binary representation of a number divisible by 3}\}$.
\end{itemize}

\section*{Problem 2}

Let $G = (V,E)$ be an unweighted, undirected graph with $n$ vertices and $m$ edges. Suppose that we do not want to find just one minimum cut, but want to count the {\em number} of minimum cuts (recall in class that we said the number of minimum cuts is never more than $\binom{n}{2}$, which is achieved by the $n$-cycle, but in general the number of minimum cuts could be any integer between $1$ and $\binom{n}{2}$). In this problem we will give a randomized algorithm to accomplish this task.
\begin{itemize}
\item[(a)] (3 points) Suppose we have $n$ colored balls in a bucket, each with a different color. At each time step, we pick a uniformly random ball, observe its color, then put it back in the bucket. Show that the expected number of time steps before we observe each color at least once is $O(n\log n)$.
\item[(b)] (7 points) Give a randomized Monte Carlo algorithm to exactly count the number of minimum cuts. You may assume that one run of the contraction algorithm, to output a single cut (which we said in class is a mincut with probability at least $1/\binom{n}{2}$), can be implemented to take time $O(n^2)$. A modified version of Karger's basic contraction algorithm to solve this problem part is sufficient to receive full credit --- you need not attempt to modify Karger-Stein. Your algorithm should fail to output the correct answer with probability at most $P$, for some given $0 < P < 1$.
\end{itemize}

\section*{Problem 3}

Let $L_k = \{w\in \{a,b\}^* : \mbox{the $k$th symbol from the end of $w$ is $a$}\}$.

\begin{enumerate}
\item[(a)] (5 points) Show that $L_k$ is recognized by a $(k+1)$-state NFA $N_3$.  Draw the state diagram of $N_3$ and apply the subset construction to $N_3$ to obtain a DFA for $L_3$.

\item[(b)] (5 points) Show that every DFA to recognize $L_k$ requires at least $2^k$ states. (Hint: use the Myhill-Nerode Theorem.)

\item[(c)] (\textbf{Challenge problem}, 0 points) The above shows that the subset construction is within a factor of 2 of optimal (since a language given by an NFA with $|Q|=k+1$ states requires at least $2^k=2^{|Q|}/2$ states as a DFA).  Close the gap between the upper bound and lower bound as much as you can.
\end{enumerate}


\section*{Problem 4}

In class, we saw how to decide whether a pattern $w\in \Sigma^*$ of length $m$ is a substring of a string $x\in \Sigma^*$ of length $n$ in time $O(m^3\cdot |\Sigma|+n)$ by constructing a DFA $M_w=(Q=\{0,\ldots,m\},\Sigma,\delta_w,q_0=0,F=\{m\})$ from $w$ and then running $M_w$ on $x$.  Here you will see how to improve the algorithm to run in time $O(m+n)$.    Given a pattern $w$, define an array $\pi_w=(\pi_w(1),\ldots,\pi_w(m))\in \{0,\ldots,m\}^{m}$ where
    $\pi_w(i)$ is defined to be the largest $j<i$ such that
    $w_1w_2\cdots w_j = w_{i-j+1}w_{i-j+2}\cdots w_i$.
\begin{itemize}
\item[(a)] (3 points) Show that given $w$, $\pi_w$, $q\in \{0,\ldots,m\}$, and $\sigma\in \Sigma$, the transition function $\delta_w(q,\sigma)$ can be evaluated in time at most $O(q+2-\delta_w(q,\sigma))$.
\item[(b)] (3 points) Show that given $w$, $\pi_w$, and a string $x\in \Sigma^*$ of length $n$, we can decide whether $w$ is a substring of $x$ in time $O(n)$.  \textbf{Hint:} use (a) and look for a telescoping sum to obtain an amortized analysis.
\item[(c)] (4 points) Show that given $w$, the array $\pi_w$ can be constructed in time $O(m)$. \textbf{Hint:} use $\pi_w(1),\ldots,\pi_w(i-1)$ to help construct $\pi_w(i)$ and again use an amortized analysis.
\end{itemize}

\section*{Problem 5 (Programming Problem)}
Solve ``FIELD'' on the programming server \url{https://cs125.seas.harvard.edu}.\\
(under ``Problem Set 6'').

\end{document}