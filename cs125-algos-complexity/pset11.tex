\documentclass[12pt]{article}

\usepackage{url}
\usepackage{fullpage}
\usepackage{amssymb,amsmath}
\usepackage{graphicx}

\newcommand{\Obj}{\mathrm{Obj}}
\newcommand{\NP}{\mathrm{NP}}
\newcommand{\PTIME}{\mathrm{P}}
\newcommand{\coNP}{\mbox{co-NP}}
\newcommand{\textprob}[1]{\textsc{#1}}

\DeclareMathOperator*{\E}{\mathbb{E}}
\let\Pr\relax
\DeclareMathOperator*{\Pr}{\mathbb{P}}

\begin{document}

\thispagestyle{empty}

\begin{center}
{\Large \textsc{CS 125 Algorithms \& Complexity} --- Fall 2016}

\bigskip

{\Large \textsc{Problem Set 11}}

\smallskip

Due: 11:59pm, Wednesday, November 30th

\bigskip

{\footnotesize See homework submission instructions at \url{http://seas.harvard.edu/~cs125/fall16/schedule.htm}}
\end{center}


\section*{Problem 1} 

If we restrict the problems we look at, sometimes hard problems like counting the number of independent sets are in a graph become solvable.

\begin{itemize}
\item[(a)] (3 points) Consider a graph that is a path on $n$ vertices.  (That is, the vertices are labelled 1 to $n$, and there is an edge from 1 to 2, 2 to 3, etc.)  How many independent sets are there as a function of $n$? We want to express your answer in terms of a family of numbers -- like ``For $n$ vertices the number of independent sets is the $n$th prime.'' (note: that is not the answer).
\item[(b)] (3 points) How many independent sets are there on a cycle of $n$ vertices?
\item[(c)] (4 points) How many independent sets are there on a complete binary tree with $127$ nodes? Describe how you arrived at this number.
\end{itemize}

\section*{Problem 2}

Consider the problem MAX-$k$-CUT, which is like the MAX CUT algorithm, except that we divide the vertices into $k$ disjoint sets, and we want to maximize the number of edges between sets.  Explain how to generalize both the randomized and the local search algorithms for MAX CUT to MAX-$k$-CUT and prove bounds on their performance.

\section*{Problem 3}

We know that that all of NP-complete reduce to each other.  It would be nice if this meant that an approximation for one NP-hard problem would lead to another.  But this is not the case.  Consider the case of \textsc{Vertex Cover}, for which we have a polynomial-time 2-approximation algorithm.

Another NP-complete optimization problem is
\textsc{Independent Set}: given a graph $G=(V,E)$, find as large a set $S\subset V$ as possible such that no two vertices $u,v\in S$ share an edge. In particular, $S\subset V$ is a vertex cover iff its complement $V-S$ is an independent set.

\begin{itemize}
\item[(a)] (4 points) Explain why applying the above equivalence to the 2-approximation for \textsc{Vertex Cover} does not yield a constant-factor approximation algorithm for \textsc{Independent Set}.
That is, show that for every constant $c\in (0,1)$, there exists a family of graphs (growing so that the number of vertices/edges grows to infinity) for which even if we obtain a 2-approximation of the minimum vertex cover, the corresponding independent set is not within a factor of $c$ of the maximum independent set.

\item[(b)] (6 points) Using the PCP Theorem and a variant of the standard NP-completeness reductions from 3-SAT, it can be shown that both \textsc{Independent Set} and \textsc{Vertex Cover} are NP-hard to approximate to within factors $1-\epsilon_1$ and $1+\epsilon_2$, respectively, for some constants $\epsilon_1,\epsilon_2>0$. Deduce from this that \textsc{Independent Set}\ is NP-hard to approximate to within any constant factor $\alpha\in (0,1)$ by, given a graph $G$, considering the graph $G_k$ with vertex set $V_k=V^k$ and edge set $E_k=\{((u_1,\ldots,u_k),(v_1,\ldots,v_k)) : \exists i\ (u_i,v_i)\in E\}$. How does the size of the maximum independent set in $G_k$ relate to that in $G$? Why doesn't the same reduction apply to \textsc{Vertex Cover}?
\end{itemize}


\section*{Problem 4}

Sometimes it is worth running an approximation algorithm to solve a problem not because the problem is $\mathbf{NP}$-hard, but because the fastest known polynomial time algorithm we have to solve isn't as fast as we would like!

Consider for example Problem 4 from Problem Set 3 of this class (optimal layout of the nodes of a trie on blocks on disk, with block size $B$). You saw in that problem set that if there is a probability distribution over queries to the leaves of an $n$-node trie, there is a dynamic programming algorithm running in time $O(nB^2)$ which finds a layout that minimizes the expected cost of a query. (In problem set 3 it was assumed that queries could to be any node, but only querying leaves is a special case since that just corresponds to instances for which internal nodes have probability 0 of being queried.)

Suppose the optimal layout achieves an expected query cost of $\mathsf{OPT}$. Give an algorithm with improved running time $O(nB)$ which finds a layout guaranteed to achieve expected query cost at most $\mathsf{OPT} + 1$. That is, on average we only have to touch one more disk block than the optimal solution. \textbf{Hint:} reduce to the case that the input trie has at most $n/B$ leaves, suffering at most one unnecessary block transfer in your reduction.


\end{document}